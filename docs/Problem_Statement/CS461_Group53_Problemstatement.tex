\documentclass[onecolumn, draftclsnofoot,10pt, compsoc]{IEEEtran}
\usepackage{graphicx}
\usepackage{url}
\usepackage{setspace}

\usepackage{geometry}
\geometry{textheight=9.5in, textwidth=7in}

% 1. Fill in these details
\def \CapstoneTeamName{		Team PHAZE}
\def \CapstoneTeamNumber{		53}
\def \GroupMemberOne{			Alessandro Lim}
\def \GroupMemberTwo{			Touku Cha}
\def \GroupMemberThree{			Jaydeep Rotithor}
\def \CapstoneProjectName{		PHAZE | IoT Weather Station}
\def \CapstoneSponsorCompany{		Openly Published Environmental Sensing Lab}
\def \CapstoneSponsorPerson{		Chet Udell}

% 2. Uncomment the appropriate line below so that the document type works
\def \DocType{		Problem Statement
				%Requirements Document
				%Technology Review
				%Design Document
				%Progress Report
				}
			
\newcommand{\NameSigPair}[1]{\par
\makebox[2.75in][r]{#1} \hfil 	\makebox[3.25in]{\makebox[2.25in]{\hrulefill} \hfill		\makebox[.75in]{\hrulefill}}
\par\vspace{-12pt} \textit{\tiny\noindent
\makebox[2.75in]{} \hfil		\makebox[3.25in]{\makebox[2.25in][r]{Signature} \hfill	\makebox[.75in][r]{Date}}}}
% 3. If the document is not to be signed, uncomment the RENEWcommand below
%\renewcommand{\NameSigPair}[1]{#1}

%%%%%%%%%%%%%%%%%%%%%%%%%%%%%%%%%%%%%%%
\begin{document}
\begin{titlepage}
    \pagenumbering{gobble}
    \begin{singlespace}
    	\includegraphics[height=4cm]{coe_v_spot1}
        \hfill 
        % 4. If you have a logo, use this includegraphics command to put it on the coversheet.
        %\includegraphics[height=4cm]{CompanyLogo}   
        \par\vspace{.2in}
        \centering
        \scshape{
            \huge CS Capstone \DocType \par
            {\large\today}\par
            \vspace{.5in}
            \textbf{\Huge\CapstoneProjectName}\par
            \vfill
            {\large Prepared for}\par
            \Huge \CapstoneSponsorCompany\par
            \vspace{5pt}
            {\Large\NameSigPair{\CapstoneSponsorPerson}\par}
            {\large Prepared by }\par
            Group\CapstoneTeamNumber\par
            % 5. comment out the line below this one if you do not wish to name your team
            \CapstoneTeamName\par 
            \vspace{5pt}
            {\Large
                \NameSigPair{\GroupMemberOne}\par
                \NameSigPair{\GroupMemberTwo}\par
                \NameSigPair{\GroupMemberThree}\par
            }
            \vspace{20pt}
        }
        \begin{abstract}
        % 6. Fill in your abstract    
        	Remote data collection is a challenging job where data needs to be collected frequently.
            The OPEnS Lab at OSU is currently using an Arduino microcontroller to collect weather information at HJ Andrews Research forest.
            The issue with the device is that it operates in a moisture rich environment not suitable for electronics and has been taken offline.
            Our project will help improve the current microcontroller and establish a network of Arduinos to act as a weather station.
            The weather station will collect various weather measurements and transmit the data wirelessly to a central hub.
            The data will be uploaded to a database and visualized on a website.
        	The result will be a comparably low-priced, open-source weather station that can be deployed in any environment for researchers around the world.
        \end{abstract}     
    \end{singlespace}
\end{titlepage}
\newpage
\pagenumbering{arabic}
\tableofcontents
% 7. uncomment this (if applicable). Consider adding a page break.
%\listoffigures
%\listoftables
\clearpage

% 8. now you write!
\section{Description}
In the past, there has been a weather-monitoring device active in the HJ Andrews Research Forest west of the Cascades. This device is very versatile, since it can capture various measurements such as temperature and humidity. Unfortunately, there are several issues with this device that need to be resolved. 
\newline
One such issue is that the device restarts abruptly, and this causes it to lose all saved data. This can be bothersome, since the device is deployed in a remote location in the H.J. Andrews Research Forest and a large number of factors can make the device go through a power cycle without warning. If power loss were to occur for any reason, all of the data stored so far in the device would be wiped out, and someone would have to travel for several hours into the forest to reset the device.
\newline
Another issue is that the device currently does not have a way of measuring the albedo effect, or reflected light. Albedo is important to measure, since it gives a lot of information on the surrounding environment and what materials are present in it. 
\newline
A third issue is that to get reliable measurements from different locations, we would need to deploy more devices in different areas of the forest. To do this we would need to create a large number of devices.  Once we get these measurements, we will need a way to derive meaning from them. We will have to find patterns in the measurements and correlate them to the condition of the environment when the measurements were taken. This will need to be done via data plotting.
\newline
Finally, the battery life of the device is not long enough. A short battery life causes the device to need maintenance more often than it should.  To replace the battery, someone would be required to travel all the way into the forest. This can be very tiring, and over time can be dangerous and wasteful of resources. It takes two hours to travel into the forest, so repeatedly traveling there can significantly increase gasoline expenditures.  Once at the device there isn’t a quick and easy way to reset the device without dismantling it.  If the device were to be deployed in an even more remote location, it would be even more difficult to access it and even more resources would be spent trying to get to it.


\section {Solution}
A new evaporometer device will be designed to fix previous flaws. The evaporometer will still measure temperature, rainfall, and humidity with the sensors used in the previous device. However, more features are to be added. 
\newline
First, to prevent data loss from power loss, memory storage will be used. The device will store its measured data in the storage. In the case of power loss we can still grab the measurement at the time it was supposed to be stored.
\newline
Second, to measure the albedo, two light sensors will be added to the device. One of the sensors will measure light intensity directly from the sun, and the other sensors will measure the light from below. Once the data is measured, the albedo effect can be calculated. 
\newline
Third, to get reliable measurements from the area, we will be deploying 20(1) devices. The devices will transmit data to a network hub at timed intervals simultaneously. The network hub will upload the data to a database, which will be visualized on a website.
\newline
Finally, to remove the need to replace the battery on the device, a solar panel will be installed to the device. It will charge a battery that powers the device when not transmitting data.

\section {Performance Metrics}
Our performance will be evaluated based on the number of Arduinos we can deploy and the Plot.ly site that we build once we have collected data. 
\newline
Our first goal is to deploy 20\footnotemark Arduinos.  Each Arduino at the minimum should be able to:
\begin{itemize}
\item Measure temperature, humidity, rainfall, solar radiation, carbon dioxide levels, wind, and light.
\item Persist for half a year\footnotemark without physical maintenance and be rechargeable via solar panels.
\item Log data to the storage device and transmit data to the network hub.
\item Transmit data to the hub simultaneously with all connected devices.
\end{itemize}
Our second goal is to improve and add-on to the source-code.  There should be improvements in:
\begin{itemize}
\item Transmitting the data, data should not be corrupt when transmitting.
\item Receiving the data, the hub should be able to check if a device did transmit data instead of accepting everything.
\item Real-world values, converting the data to scientific values readable by researchers.
\end{itemize}
Our final goal is to build a website to host all the information collected, it should contain:
\begin{itemize}
\item Visuals such as charts and graphs to help researchers view information remotely.
\item Pre-sorted data to create new visuals.
\item Links to the database to easily change values over time.
\end{itemize}
\footnotetext[1]{Subject to change, our meeting with the team indicated that only 5 was possible.}
\footnotetext[2]{With the school year we won’t be able to test for this, but we can calculate it in the lab once everything is setup.
}





\end{document}