\documentclass[onecolumn, draftclsnofoot,10pt, compsoc]{IEEEtran}
\usepackage{graphicx}
\usepackage{url}
\usepackage{setspace}
\usepackage{pgfgantt}


\usepackage{geometry}
\geometry{textheight=9.5in, textwidth=7in}
\title{Progress Report}
% 1. Fill in these details
\def \CapstoneTeamName{		Team PHAZE}
\def \CapstoneTeamNumber{		53}
\def \GroupMemberOne{			Alessandro Lim}
\def \GroupMemberTwo{			Touku Cha}
\def \GroupMemberThree{			Jaydeep Rotithor}
\def \CapstoneProjectName{		PHAZE | IoT Weather Station}
\def \CapstoneSponsorCompany{		Openly Published Environmental Sensing Lab}
\def \CapstoneSponsorPerson{		Chet Udell}

% 2. Uncomment the appropriate line below so that the document type works
\def \DocType{		%Problem Statement
				%Requirements Document
				%Technology Review
				%Design Document
				Progress Report
				}
			
\newcommand{\NameSigPair}[1]{\par
\makebox[2.75in][r]{#1} \hfil 	\makebox[3.25in]{\makebox[2.25in]{\hrulefill} \hfill		\makebox[.75in]{\hrulefill}}
\par\vspace{-12pt} \textit{\tiny\noindent
\makebox[2.75in]{} \hfil		\makebox[3.25in]{\makebox[2.25in][r]{Signature} \hfill	\makebox[.75in][r]{Date}}}}
% 3. If the document is not to be signed, uncomment the RENEWcommand below
%\renewcommand{\NameSigPair}[1]{#1}

%%%%%%%%%%%%%%%%%%%%%%%%%%%%%%%%%%%%%%%
\begin{document}
\begin{titlepage}
    \pagenumbering{gobble}
    \begin{singlespace}
    	\includegraphics[height=4cm]{coe_v_spot1}
        \hfill 
        % 4. If you have a logo, use this includegraphics command to put it on the coversheet.
        %\includegraphics[height=4cm]{CompanyLogo}   
        \par\vspace{.2in}
        \centering
        \scshape{
            \huge CS Capstone \DocType \par
            {\large\today}\par
            \vspace{.5in}
            \textbf{\Huge\CapstoneProjectName}\par
            \vfill
            {\large Prepared for}\par
            \Huge \CapstoneSponsorCompany\par
            \vspace{5pt}
            {\Large\NameSigPair{\CapstoneSponsorPerson}\par}
            {\large Prepared by }\par
            Group\CapstoneTeamNumber\par
            % 5. comment out the line below this one if you do not wish to name your team
            \CapstoneTeamName\par 
            \vspace{5pt}
            {\Large
                \NameSigPair{\GroupMemberOne}\par
                \NameSigPair{\GroupMemberTwo}\par
                \NameSigPair{\GroupMemberThree}\par
            }
            \vspace{20pt}
        }
        \begin{abstract}
                % 6. Fill in your abstract   
                This report will recap our progress during the Fall term of the Evaporometer project.
                We summarize the 10 weeks we have been working on this project and in each week we discuss our plans for that week, the problems we faced, and our progress.

                
        \end{abstract}     
    \end{singlespace}
\end{titlepage}

\newpage
\pagenumbering{arabic}
\tableofcontents
% 7. uncomment this (if applicable). Consider adding a page break.
%\listoffigures
%\listoftables
\clearpage

\section{Project Purpose and Goals}
The purpose of this project is to create a device that gathers information about humidity, temperature, rainfall, and albedo effect. This project also works on the visualization of the measured data.

The goal of this project is to deploy 20 of the device in H.J. Andrews Forest and create a visualization website of the measured data.

\section{Where We are Currently At}
So far, we have worked on getting an idea of what we will be working on for the rest of the year. We have worked on several modifications to the Evaporometer devices, and we have also begun work on the web mapping portion of the project, which will be used to model the data that is collected from the device. Over the past several weeks, we have worked on web mapping tutorials to get us up to speed on the tools that we will need. We have also been informed that we will need to create multiple Evaporometer devices to deploy in several places across the rainforest.

\section{Weekly Progress}
\subsection{Week 3}
\subsubsection{Jaydeep}
Plans: Meet with several people who will help guide us in the right direction with regards to the project: Manuel Lopez, Tom DeBell, and Marissa Kwon. 
\newline
Problems: While we did meet with Chet Friday afternoon, all we got was a big picture view of the project and we still need to figure out where to start and what needs to be improved on.
\newline
Progress:   We were able to gain some direction over the week in terms of where to start on the project and what things we would be working on. 

\subsubsection{Touku}
Plans: Schedule time to meet with the OPENs team regarding the project.
\newline
Problems: Only met with Chet and received a small bits of information about the project
\newline
Progress: We will be getting in contact with the rest of the OPENs team who worked on this last year so we’ll have a better understanding of the project and where to proceed.

\subsubsection{Alessandro}
Plans: Set up meetings with Manuel, Tom, and Marissa. Join the project github page. Find about integrating solar panel, and non volatile memory into arduino. 
\newline
Problems: Have to read Arduino Cookbook
\newline
Progress: We met with Chet

\subsection{Week 4}
\subsubsection{Jaydeep}
Plans: Get a better idea of what we will be working on over the course of the year by having a team meeting.
\newline
Problems: We have met with several team members, but we do not yet have a full idea of the problems that we need to solve. 
\newline
Progress:  We met with the team and were able to get a rundown of the device that we will be working on and what parts of it we will be improving. 

\subsubsection{Touku}
Plans: Finalize the problem statement and meet the OPENs team.
\newline
Problems: Needed our problem statement to be looked at but Chet was not present at the meeting, although his team will look it over for us and let us know if we need to change anything.
\newline
Progress: Met team and received the hardware we’ll be using in this project, basecamp tasks to learn about the hardware.

\subsubsection{Alessandro}
Plans:  Meeting with OPENs Lab on thursday and TA on Wednesday.
\newline
Problems: None
\newline
Progress: We researched about the solar panel.

\subsection{Week 5}
\subsubsection{Jaydeep}
Plans:  Have a set of requirements against which to evaluate our final product. 
\newline
Problems: We had not met with our client the previous week, and as a result, even though we had an idea of what to do for our final product, we did not yet have a clear picture. 
\newline
Progress:  We were able to determine a set of rough guidelines for our product and created a requirements document. 

\subsubsection{Touku}
Plans: Meet with OPENs team and figure out plans for the rest of the term.
\newline
Problems: None
\newline
Progress: Met with the team, assigned task on Basecamp to accomplish by our next meeting.  Tasks involve working with the code to gain familiarity with the project.

\subsubsection{Alessandro}
Plans: Meeting with TA. Meeting with OPENs Lab team. Read more and work on EEPROM, Millis, and transmission.
\newline
Problems: Very busy week.
\newline
Progress: I found out that the feather has virtual EEPROM made with flash memory. We probably don't need to buy a standalone EEPROM.

\subsection{Week 6}
\subsubsection{Jaydeep}
Plans:   Work on several project tasks, such as:  beginning work on the restart functionality of the microcontroller that we plan to use, and working  on setting up communication between the transmitter and receiver.   
\newline
Problems: We need to set up the restart button such that it only restarts when prompted to and does not lose critical saved data. Our device currently does not have this functionality.
\newline
Progress:   We got done with several of the tasks assigned to us two weeks ago, which involved setting up the transmitter and receiver. We were assigned a new set of tasks and began to work on those. 
\subsubsection{Touku}
Plans: Finish the tutorials Thomas gave us regarding the transmitter code along with the Ethernet FeatherWing.
\newline
Problems: Thomas didn’t give me the MAC address for the ethernet so I was only able to do the tutorial for transmission.
\newline
Progress: Transmission tutorial done, need to meet up with Thomas for the MAC address.

\subsubsection{Alessandro}
Plans: Meet with the OPENs Lab and Chet. Work on Millis function. Work on receiver code.
\newline
Problems: None.
\newline
Progress:  Read about the millis function; we won’t be able to use it to complete the task. Sample receiver code done.

\subsection{Week 7}
\subsubsection{Jaydeep}
Plans: Meet with professor Bo Zhao and get a high-level understanding of the geovisualization project. Then begin work on it and go through tutorials on web mapping. 
\newline
Problems: We were informed by members of the PHAZE team that one of our members would be chosen to work on geovisualization and web mapping. We needed to have the member get familiar with the necessary tools. 
\newline
Progress: I met with Bo Zhao and was able to get a good idea of the direction that I was headed in. The team also got done with the tasks assigned, such as tare functionality and setting up the spreadsheet to handle incoming data. 

\subsubsection{Touku}
Plans: Outline tech review and continue working on tasks
\newline
Problems: None
\newline
Progress: Outline created on Google docs, starting researching technologies for project.

\subsubsection{Alessandro}
Plans: Finish Taring code sample. Meeting with Bo Zhao. Finish requirements doc as soon as we meet Bo Zhao.
\newline
Problems: Our time might be a bit limited since requirements doc is due the day after we meet Bo Zhao.
\newline
Progress: We met with Bo and finished our requirements doc

\subsection{Week 8}
\subsubsection{Jaydeep}
Plans:  This week we plan to finish the millis() function and make progress on the tutorials for web mapping.
\newline
Problems:  We need to find the averages of the data collected using the spreadsheet. Also, regarding the web mapping tutorials, I have hit a roadblock on Node.js that causes compiler errors.
\newline
Progress:   We were able to finish the millis() function and got through many of the tutorials, but because of the previously mentioned roadblock, work will need to be done on this over the weekend.

\subsubsection{Touku}
Plans: Finish the ethernet tutorial from Thomas and the rough draft of the tech review, now individual.
\newline
Problems: We choose IDE as one of our techs.  That will need to be changed but it was assigned to Jaydeep and Alessandro, will see if I can help them with it.
\newline
Progress: Proofread tech review for final draft, basecamp tasks.

\subsubsection{Alessandro}
Plans: 	Tech Review Rough Draft. Finish Timing.
\newline
Problems: The Timing requires Timer interrupt to be used. Need to read the 32u4 Atmel document to figure out how to use TCNT0.
\newline
Progress :The proof of concept of the timing interrupt is now working.
The code works for 32u4. Tweaking may be required to port to code to Feather M0.

\subsection{Week 9}
\subsubsection{Jaydeep}
Plans:  Meet with professor Bo Zhao and get headed in the right direction on the geovisualization portion of the project. 
\newline
Problems: I had trouble adding markers to maps in some of the tutorials and in the practice lab, the colors did not show up. 
\newline
Progress:  I met with professor Bo Zhao and had an idea of my next steps as well as how to begin brainstorming for the project. 

\subsubsection{Touku}
Plans: Basecamp tasks and tech review final
\newline
Problems: None
\newline
Progress: Didn’t accomplish much, most of the OPENs team left for Thanksgiving already so just going over the tasks we were assigned last week.

\subsubsection{Alessandro}
Plans: Finish tech review. No meetings this week. Do the average sampling.
\newline
Problems:None.
\newline
Progress:We finished tech review doc. I did the average sampling code.  Didn’t do much since it’s thanksgivings.


\subsection{Week 10}
\subsubsection{Jaydeep}
Plans: Finish design document, make progress on real-time web mapping tutorials, and  begin work on the progress report 
\newline
Problems: We ran into some roadblocks on the design document. We were unsure about where to place some of our technologies in the document. Also, due to exams, I was unable to make much progress on the tutorials.
\newline
Progress: We were able to complete the design document. We also got a better idea of the direction of the project for next term. 

\subsubsection{Touku}
Plans: Design document and prepare progress report.  Meet with team to record the presentation.
\newline
Problems: None
\newline
Progress: Design document submitted, progress report should be done by the Monday of Final’s week.  Need to edit presentation as it is still too short.

\subsubsection{Alessandro}
Plans: 	Meeting with TA on Tuesday at 11. Meeting with team on Thursday at 12. Finish Design Doc. Finish Progress Report.
\newline
Problems: Design doc is harder than we expected. 
\newline
Progress: Finished Design doc and did the presentation for progress report. 

\section{Retrospective}
\begin{tabular}{ |p{0.3\linewidth}|p{0.3\linewidth}|p{0.3\linewidth}|  }
\hline
Positives & Deltas & Actions\\
\hline
We were able to learn how to use the necessary tools for the web mapping portion of the project. & We will need to create a real-time web map to model our data. & We will use various web development tools along with socket.io to implement our changes. \\
\hline
We were able to get some direction on web mapping. & We need to brainstorm how to create the web map and what kind of map we need to create.& We will research the various types of maps, evaluate their pros and cons, and then see which map best suits our needs.\\
\hline
Thomas gave us enough task to accomplish by Winter term so we have a running start once we meet again.& Need to create a bill of materials for Chet and Professor Bo for the solar charging.& Need to do more research on the parts that will be needed for solar charging.\\
\hline
\end{tabular}

\end{document}